\documentclass{article}

\usepackage{fancyhdr}
\usepackage{extramarks}
\usepackage{amsmath}
\usepackage{amsthm}
\usepackage{amsfonts}
\usepackage{tikz}
\usepackage[plain]{algorithm}
\usepackage{algpseudocode}
\usepackage{enumerate}

\usepackage{listings}
\usepackage{xcolor}
\usepackage{forest}
\usepackage[shortlabels]{enumitem}
     \setlist[enumerate, 1]{1\textsuperscript{o}}
\lstset { %
    language=C++,
    backgroundcolor=\color{black!5}, % set backgroundcolor
    basicstyle=\footnotesize,% basic font setting
}

%\usetikzlibrary{automata,positioning}
\usetikzlibrary{positioning,shapes,shadows,arrows,automata}

%
% Basic Document Settings
%

\topmargin=-0.45in
\evensidemargin=0in
\oddsidemargin=0in
\textwidth=6.5in
\textheight=9.0in
\headsep=0.25in

\linespread{1.1}

\pagestyle{fancy}
\lhead{\hmwkAuthorName}
\chead{(\hmwkClassInstructor\ \hmwkClassTime): \hmwkTitle}
\rhead{\firstxmark}
\lfoot{\lastxmark}
\cfoot{\thepage}

\renewcommand\headrulewidth{0.4pt}
\renewcommand\footrulewidth{0.4pt}

\setlength\parindent{0pt}

%
% Create Problem Sections
%

\newcommand{\enterProblemHeader}[1]{
    \nobreak\extramarks{}{Problem \arabic{#1} continued on next page\ldots}\nobreak{}
    \nobreak\extramarks{Problem \arabic{#1} (continued)}{Problem \arabic{#1} continued on next page\ldots}\nobreak{}
}

\newcommand{\exitProblemHeader}[1]{
    \nobreak\extramarks{Problem \arabic{#1} (continued)}{Problem \arabic{#1} continued on next page\ldots}\nobreak{}
    \stepcounter{#1}
    \nobreak\extramarks{Problem \arabic{#1}}{}\nobreak{}
}

\setcounter{secnumdepth}{0}
\newcounter{partCounter}
\newcounter{homeworkProblemCounter}
\setcounter{homeworkProblemCounter}{1}
\nobreak\extramarks{Problem \arabic{homeworkProblemCounter}}{}\nobreak{}

%
% Homework Problem Environment
%
% This environment takes an optional argument. When given, it will adjust the
% problem counter. This is useful for when the problems given for your
% assignment aren't sequential. See the last 3 problems of this template for an
% example.
%
\newenvironment{homeworkProblem}[1][-1]{
    \ifnum#1>0
        \setcounter{homeworkProblemCounter}{#1}
    \fi
    \section{Problem \arabic{homeworkProblemCounter}}
    \setcounter{partCounter}{1}
    \enterProblemHeader{homeworkProblemCounter}
}{
    \exitProblemHeader{homeworkProblemCounter}
}

%
% Homework Details
%   - Title
%   - Due date
%   - Class
%   - Section/Time
%   - Instructor
%   - Author
%

\newcommand{\hmwkTitle}{Homework\ \#2}
\newcommand{\hmwkDueDate}{October 7, 2015}
\newcommand{\hmwkClass}{CS510 Intro to Multimedia Networking}
\newcommand{\hmwkClassTime}{Fall 2015}
\newcommand{\hmwkClassInstructor}{Wu-chi Feng}
\newcommand{\hmwkAuthorName}{Konstantin Macarenco}

%
% Title Page
%

\title{
    \vspace{2in}
    \textmd{\textbf{\hmwkClass:\ \hmwkTitle}}\\
    \normalsize\vspace{0.1in}\small{Due\ on\ \hmwkDueDate\ at 8:00am}\\
    \vspace{0.1in}\large{\textit{\hmwkClassInstructor\ \hmwkClassTime}}
    \vspace{3in}
}

\author{\textbf{\hmwkAuthorName}}
\date{}

\renewcommand{\part}[1]{\textbf{\large Part \Alph{partCounter}}\stepcounter{partCounter}\\}

%
% Various Helper Commands
%

% Useful for algorithms
\newcommand{\alg}[1]{\textsc{\bfseries \footnotesize #1}}

% For derivatives
\newcommand{\deriv}[1]{\frac{\mathrm{d}}{\mathrm{d}x} (#1)}

% For partial derivatives
\newcommand{\pderiv}[2]{\frac{\partial}{\partial #1} (#2)}

% Integral dx
\newcommand{\dx}{\mathrm{d}x}

% Alias for the Solution section header
\newcommand{\solution}{\textbf{\large Solution}}

% Probability commands: Expectation, Variance, Covariance, Bias
\newcommand{\E}{\mathrm{E}}
\newcommand{\Var}{\mathrm{Var}}
\newcommand{\Cov}{\mathrm{Cov}}
\newcommand{\Bias}{\mathrm{Bias}}

\begin{document}

\maketitle

\pagebreak

\begin{homeworkProblem}
Suppose we are using Huffman Compression on the following symbols and their
probabilities. What is the expected compression ratio (for a randomly generated
sequence with the same probabilities), assuming that the original symbols are 8-bits
in length.\\

Knowing the distribution we can estimate an approximate size of encoded string.
First we need to build the encoding tree.

\begin{forest}
for tree={
  draw,
  minimum height=1cm,
  anchor=north,
  align=center,
  grow=north,
  child anchor=south
},
[{1}, tier=1
    [{0.4}, tier=2, edge label={node[midway,below,font=\scriptsize]{1}} 
        [{0.2}, tier=3,  edge label={node[midway,left,font=\scriptsize]{1}}
            [{B, 0.10}, tier=4,  edge label={node[midway,left,font=\scriptsize]{1}}]
            [{D, 0.10}, tier=4,  edge label={node[midway,left,font=\scriptsize]{0}}]
        ]
        [{A, 0.20}, tier=4, edge label={node[midway,left,font=\scriptsize]{0}} ]
    ]
    [{0.6}, tier=2, edge label={node[midway,below,font=\scriptsize]{0}} 
        [{0.3}, tier=3, edge label={node[midway,auto,font=\scriptsize]{1}}
            [{F, 0.15}, tier=4, edge label={node[midway,left,font=\scriptsize]{1}} ]
            [{C, 0.15}, tier=4, edge label={node[midway,left,font=\scriptsize]{0}} ]
        ]
        [{E, 0.30}, tier=4, edge label={node[midway,auto,font=\scriptsize]{0}} ]
    ]
]
% \node[anchor=west,align=left] 
%   at ([xshift=-2cm]MS.west) {Level 3\\Criteria};
% \node[anchor=west,align=left] 
%   at ([xshift=-2cm]MS.west|-PDC) {Level 2\\ Group Criteria};
% \node[anchor=west,align=left] 
%   at ([xshift=-2cm]MS.west|-SS) {Level 1\\Overall Objective};
\end{forest}

\begin{itemize}
\item A 10
\item B 111
\item C 010
\item D 110
\item E 00
\item F 011
\end{itemize}

Considering a random text that contains 100 characters, each character 8bits $100*8 = 800$ bits in total.
With Huffman compression we get $20*2 + 10*3 + 15 *3 + 10*3 + 30*2 + 15*3 = 250$ bits,
so compression ratio \textbf{$3.2 : 1$}, or around 70\%.

However this calculation doesn't account for the size of encoding table.
    
\end{homeworkProblem}

\begin{homeworkProblem}
USB1 bandwidth is approximately 11 megabits per second. Suppose we have a
camera that we have attached to our computer that is capable of capturing 640x480
pixel video at 30 fps.  \\
\begin{enumerate}[(a), leftmargin = 0.7cm, nosep]
\item What is the maximum frame rate that we can achieve over this channel? \\
   frame size = $640 \times 480 \times 3 = 921600 $ bits\\
   Bandwidth  = $11 \times 10^6 = 11000000$ bits\\
   Maximum fps = $\cfrac{Bandwidth}{frame size} = \cfrac{11000000}{921600} \approx 11.93576$\\
   \textbf{Answer: maximum frame rate $ \approx 11.93576$ fps }\\
\item  What compression ratio would we need to achieve 30fps? \\
    One frame size at 30 fps is $640 \times 480 \times 3 \times 30 = 27648000$ \\
    To be able to transmit it over USB we have to compress it to $\cfrac{27648000}{11000000} \approx 2.514$ times.\\
   \textbf{Answer: 2.514 times}\\
\item  What is the maximum sized 4:3 aspect ratio video that can be captured over the
USB channel?\\
   Maximum number of pixels we can translate over the given USB = $\cfrac{11000000}{30 \times 3} \approx 122222.222 $ \\
   where $x \times y = 122222.222$ \text{ and } $\cfrac{4}{3} = \cfrac{x}{y},$ \\
   Solving for : $x = \cfrac{4}{3}\times y$ \\
   $\cfrac{4}{3}\times y \times y = 122222.222$ \\
   $\cfrac{4}{3}\times y^2 = 122222.222$ \\
   $y^2 = \cfrac{3}{4}\times 122222.222$ \\
   $y^2 = \cfrac{3}{4}\times 122222.222$ \\
   $y^2 = 91666.666$ \\
   $y = \sqrt{91666.666}$ \\
   $y \approx 302.765 $ \\
   
   And $x = \cfrac{302.765 * 4}{3}\ \approx 403.686$ \\
   \textbf{Answer: Max sized aspect $\approx 403:302$ }\\
\end{enumerate}
    
\end{homeworkProblem}

\begin{homeworkProblem}

A B C A C A B C B A B

\begin{table}[h!]
    \centering
    \begin{tabular}{|c|c|c|c|}
    \hline
     w    &   k   &   dictionary entries & output     \\ \hline
          &   A   &                      &            \\ \hline
     A    &   B   &    $<256>$ AB        & A          \\ \hline
     B    &   C   &    $<257>$ BC        & B          \\ \hline
     C    &   A   &    $<258>$ CA        & C          \\ \hline
     A    &   C   &    $<259>$ AC        & A          \\ \hline
     C    &   A   &    exists            &            \\ \hline
     CA   &   B   &    $<260> $ CAB      & $<258>$    \\ \hline
     B    &   C   &    exists            &            \\ \hline
     BC   &   B   &    $<261>$  BCB      & $<257>$    \\ \hline
     B    &   A   &    $<262>$  BA       &  B         \\ \hline
     A    &   B   &    exists            &            \\ \hline
     AB   &       &                      &  AB        \\ \hline
    \end{tabular}
\end{table}

    
\end{homeworkProblem}

\begin{homeworkProblem}

Decompress: \\ 
F A B \_ L $<257>$ \_ C $<261>$ $<264>$ T $<257>$

\begin{table}[h!]
    \centering
    \begin{tabular}{|c|c|c|c|c|}
    \hline
     w    &   k         & entry  &   dictionary         & output     \\ \hline
     F    &   F         &  F     &                      &  F         \\ \hline
     F    &   A         &  A     &   $<256>$ FA         &  A         \\ \hline
     A    &   B         &  B     &   $<257>$ AB         &  B         \\ \hline
     B    &  \_         & \_     &   $<258>$ B\_        &  \_        \\ \hline
    \_    &   L         &  L     &   $<259>$ \_L        &  L         \\ \hline
     L    & $<257>$     & AB     &   $<260>$ LA         &  AB        \\ \hline
    AB    &  \_         & \_     &   $<261>$ AB\_       &  \_        \\ \hline
    \_    &   C         &  C     &   $<262>$ \_L        &  C         \\ \hline
     C    & $<261>$     & AB\_   &   $<263>$ CA         &  AB\_      \\ \hline
    AB\_  & $<264>$     & AB\_A  &   $<264>$ AB\_A      &  AB\_A     \\ \hline
  AB\_A   &   T         &  T     &   $<265>$ AB\_AT     &  T         \\ \hline
     T    & $<257>$     & AB     &   $<266>$ TA         &  AB        \\ \hline
    \end{tabular}
\end{table}
   
\textbf{Output string: FAB\_LAB\_CAB\_AB\_ATAB }
\end{homeworkProblem}

\end{document}
