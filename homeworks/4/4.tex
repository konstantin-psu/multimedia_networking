\documentclass{article}

\usepackage{fancyhdr}
\usepackage{extramarks}
\usepackage{amsmath}
\usepackage{amsthm}
\usepackage{amsfonts}
\usepackage{tikz}
\usepackage[plain]{algorithm}
\usepackage{algpseudocode}
\usepackage{enumerate}

\usepackage{listings}
\usepackage{xcolor}
\usepackage{forest}
\usepackage[shortlabels]{enumitem}
     \setlist[enumerate, 1]{1\textsuperscript{o}}
\lstset { %
    language=C++,
    backgroundcolor=\color{black!5}, % set backgroundcolor
    basicstyle=\footnotesize,% basic font setting
}

%\usetikzlibrary{automata,positioning}
\usetikzlibrary{positioning,shapes,shadows,arrows,automata}

%
% Basic Document Settings
%

\topmargin=-0.45in
\evensidemargin=0in
\oddsidemargin=0in
\textwidth=6.5in
\textheight=9.0in
\headsep=0.25in

\linespread{1.1}

\pagestyle{fancy}
\lhead{\hmwkAuthorName}
\chead{(\hmwkClassInstructor\ \hmwkClassTime): \hmwkTitle}
\rhead{\firstxmark}
\lfoot{\lastxmark}
\cfoot{\thepage}

\renewcommand\headrulewidth{0.4pt}
\renewcommand\footrulewidth{0.4pt}

\setlength\parindent{0pt}

%
% Create Problem Sections
%

\newcommand{\enterProblemHeader}[1]{
    \nobreak\extramarks{}{Problem \arabic{#1} continued on next page\ldots}\nobreak{}
    \nobreak\extramarks{Problem \arabic{#1} (continued)}{Problem \arabic{#1} continued on next page\ldots}\nobreak{}
}

\newcommand{\exitProblemHeader}[1]{
    \nobreak\extramarks{Problem \arabic{#1} (continued)}{Problem \arabic{#1} continued on next page\ldots}\nobreak{}
    \stepcounter{#1}
    \nobreak\extramarks{Problem \arabic{#1}}{}\nobreak{}
}

\setcounter{secnumdepth}{0}
\newcounter{partCounter}
\newcounter{homeworkProblemCounter}
\setcounter{homeworkProblemCounter}{1}
\nobreak\extramarks{Problem \arabic{homeworkProblemCounter}}{}\nobreak{}

%
% Homework Problem Environment
%
% This environment takes an optional argument. When given, it will adjust the
% problem counter. This is useful for when the problems given for your
% assignment aren't sequential. See the last 3 problems of this template for an
% example.
%
\newenvironment{homeworkProblem}[1][-1]{
    \ifnum#1>0
        \setcounter{homeworkProblemCounter}{#1}
    \fi
    \section{Problem \arabic{homeworkProblemCounter}}
    \setcounter{partCounter}{1}
    \enterProblemHeader{homeworkProblemCounter}
}{
    \exitProblemHeader{homeworkProblemCounter}
}

%
% Homework Details
%   - Title
%   - Due date
%   - Class
%   - Section/Time
%   - Instructor
%   - Author
%

\newcommand{\hmwkTitle}{Homework\ \#4}
\newcommand{\hmwkDueDate}{October 28, 2015}
\newcommand{\hmwkClass}{CS510 Intro to Multimedia Networking}
\newcommand{\hmwkClassTime}{Fall 2015}
\newcommand{\hmwkClassInstructor}{Wu-chi Feng}
\newcommand{\hmwkAuthorName}{Konstantin Macarenco}

%
% Title Page
%

\title{
    \vspace{2in}
    \textmd{\textbf{\hmwkClass:\ \hmwkTitle}}\\
    \normalsize\vspace{0.1in}\small{Due\ on\ \hmwkDueDate\ at 8:00am}\\
    \vspace{0.1in}\large{\textit{\hmwkClassInstructor\ \hmwkClassTime}}
    \vspace{3in}
}

\author{\textbf{\hmwkAuthorName}}
\date{}

\renewcommand{\part}[1]{\textbf{\large Part \Alph{partCounter}}\stepcounter{partCounter}\\}

%
% Various Helper Commands
%

% Useful for algorithms
\newcommand{\alg}[1]{\textsc{\bfseries \footnotesize #1}}

% For derivatives
\newcommand{\deriv}[1]{\frac{\mathrm{d}}{\mathrm{d}x} (#1)}

% For partial derivatives
\newcommand{\pderiv}[2]{\frac{\partial}{\partial #1} (#2)}

% Integral dx
\newcommand{\dx}{\mathrm{d}x}

% Alias for the Solution section header
\newcommand{\solution}{\textbf{\large Solution}}

% Probability commands: Expectation, Variance, Covariance, Bias
\newcommand{\E}{\mathrm{E}}
\newcommand{\Var}{\mathrm{Var}}
\newcommand{\Cov}{\mathrm{Cov}}
\newcommand{\Bias}{\mathrm{Bias}}

\begin{document}

\maketitle

\pagebreak

%\begin{enumerate}[(a), leftmargin = 0.7cm, nosep]
\begin{homeworkProblem}

What are four main differences between GIF and PNG?
\begin{enumerate}[1), leftmargin = 0.9cm, nosep]
\item Color Pallet. GIF uses only 256 colors pallet, when JPEG is full color (24 bit)
\item Encoding. GIF uses rather simplistic encoding, like LZW, when creating JPEG images is fairly complicated.
\item Purpose - GIF great for images that use small set of colors for example logo.
\item Tunable compression rate. JPEG images quality can be adjusted, whereas GIF cannot.
\end{enumerate}

\end{homeworkProblem}

\begin{homeworkProblem}

Why is zig-zag reordering used in JPEG compression?

Resulting matrix after quantization tend to contain most of the coefficients clustered together in the upper left corner.
Zig-zag reordering helps to keep most of the zero coefficients together.

\end{homeworkProblem}

\begin{homeworkProblem}

What are the entropy encoding techniques employed in JPEG? \\

JPEG uses all encodings studied so far: run length (RLE), Huffman, LZW, zig-zag reordering.

\end{homeworkProblem}
\begin{homeworkProblem}

What is the main difference between how the macroblocks are encoded in h.261 versus MPEG-1? \\

The main difference is that MPEG-1 in addition to I and P frames can contain B frames - motion based bidirectional predictive coding frame, which uses
both past reference frames and future reference frames, which allows higher compression rate.

\end{homeworkProblem}
\begin{homeworkProblem}
Suppose we have a set of frames (number 1 ... n) that we would like to compress into MPEG-1 with the following frame types:\\
$I_0\text{ }B_1\text{ }B_2\text{ }I_3\text{ }P_4\text{ }B_5\text{ }P_6\text{ }I_7\text{ }B_8\text{ }B_9\text{ }I_{10}\text{ }I_{11}\text{ }B_{12}\text{ }B_{13}\text{ }P_{14}\text{ }P_{15}\text{ }P_{16}\text{ }I_{17}$\\
Assuming, they all fit into one group of pictures (GOP), what order do these frames appear in the compressed stream?\\

Frames should be reordered so each frame that depends on another frame should appear after it in the compressed file.

$I_0\text{ }I_3\text{ }B_1\text{ }B_2\text{ }P_4\text{ }P_6\text{ }B_5\text{ }I_7\text{ }I_{10}\text{ }B_8\text{ }B_9\text{ }I_{11}\text{ }P_{14}\text{ }B_{12}\text{ }B_{13}\text{ }P_{15}\text{ }P_{16}\text{ }I_{17}$\\

\end{homeworkProblem}

\end{document}